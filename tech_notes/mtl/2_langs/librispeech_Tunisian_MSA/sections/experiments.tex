\section{EXPERIMENT}
\label{sec-4}

We used the kaldi toolkit\cite{Povey11thekaldi} to build our \gls{asr} systems. 
We derived our setup from the kaldi babel multilang recipe. 

a \gls{ann} was built with two kinds of layers.
\begin{enumerate}
\item Shared Layers.
\item Language Specific Layers.
\end{enumerate}

The shared layers were trained on all the training data from both the \gls{tl} and \gls{bl} languages.

The language specific layers were trained only on data from the \gls{tl}.

We  tried to write our recipe so that  it only contain  steps that are required to implement the \gls{mtl} method. 
Thus, We did not use i-vectors which are standard in many kaldi recipes. 
however, we did include  a bottleneck layer. 

We built baseline \gls{sat} \gls{gmm} \gls{hmm} \glspl{am} for both the Librispeech and Tunisian \gls{msa} corpora.
For the Tunisian \gls{msa} baseline system we derived our pronouncing dictionary from the 2 million entries from the \gls{qcri} vowelized dictionary\cite{aaliArabicKaldi} available at:

\url{http://alt.qcri.org/resources/speech/dictionary/ar-ar_lexicon_2014-03-17.txt.bz2}

We added the \gls{oov} words from the Tunisian \gls{msa} training set and the test set. 
We trained our $3$-gram language model with the \gls{srilm} toolkit\cite{Stolcke02srilm-} on the transcripts from the training and test data. 
Our best \gls{wer} results for Tunisian \gls{msa} were obtained with online chain models.

We followed the kaldi standard recipe for the librispeech cleaned 960 hours of speech task with one exception. 
We extracted and trained with \gls{plp} pitch features instead of \gls{mfcc} features. 
We followed this path since we derived our scripts from the kaldi babel multilang recipe. 
In the future We plan on incorporating tonal \glspl{bl} and we expect better results using \gls{plp} pitch instead of \gls{mfcc} features. 

\subsection{\gls{ann} Configuration}
\label{sec:nnconfig}


The \gls{ann} had ten layers.

\begin{enumerate}
\item One input layer,
\item 6 hidden layers,
\item One Bottleneck layer,
\item One affine layer, and
\item One soft max layer.
\end{enumerate}


The \gls{relu} function was used to compute activations.
The dimension of the hidden layers was 1024.
The dimension of the Bottleneck layer was 512.
The final layer implemented  a soft max function that output a probability density function over the clustered triphones.
The frame Context was set to 16 frames to the left and 12 frames to the right.

\subsection{\gls{am} Training}
\label{sec:train}


A bilingual raw feedforward deep \gls{ann}  was trained on the combined set of training examples from the English Librispeech and Tunisian \gls{msa}  corpora using a frame-level cross-entropy objective function.
The data from the Tunisian \gls{msa} corpus was used to readjust the parameters in the last two layers of the bilingual \gls{dnn} model to produce a new monolingual Tunisian \gls{msa}  acoustic model. 
Similarly, a new monolingual English model was produced.
These two models shared the parameters in their first eight layers, only their final 2 layers were different.

\subsection{Decoding}
\label{sec:decode}


The monolingual system with the Tunisian \gls{msa} \gls{am} was used to decode a test set of speech from four speakers, 3 Libyan males and one Tunisian female.
The same \gls{fst} decoding graph that was built for the Tunisian \gls{msa} \gls{sat} \gls{gmm} \gls{hmm} system was used for decoding with the \gls{mtl} \gls{am} set.
